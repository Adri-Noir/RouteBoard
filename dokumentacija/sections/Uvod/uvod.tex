\chapter{Uvod}

Digitalna tehnologija obuhaća gotovo sve aspekte ljudskog života, od komunikacije, poslovanja do zabave i znanja. Rekreativne aktivnosti i sportovi koji su se oslanjali na fizičku opremu i materijale sve više usvajaju digitalne alate koji proširuju mogućnosti i količinu informacija koje korisnici mogu dobiti. Sportsko penjanje, kao aktivnost koja spaja fizičku spremnost i boravak u prirodi, predstavlja primjer aktivnosti koja se može proširiti digitalnim alatima.

Sportsko penjanje i srodna disciplina \textit{bouldering} posljednjih su desetljeća doživjeli eksponencijalni rast u popularnosti, privlačeći sve veći broj zainteresiranih ljudi kako u specijalizirane penjačke dvorane, tako i na stijene u prirodi. Na Olimpijskim igrama 2020. godine u Tokiju sportsko penjanje je po prvi put uvršten u program čime je sport dobio globalnu pozornost i dodatno potaknuo interes javnosti. Olimpijskim igrama 2024. godine u Parizu popularnost sporta je još više porasla. Prema članku iz \textit{The Oxford Blue}, dok se vrijednost globalnog tržišta penjačkih dvorana procjenjuje na 117.61 milijardi dolara do 2031. godine \cite{the_oxford_blue_rock_climb}. S rastom zajednice, raste i potreba za kvalitetnim, dostupnim i preciznim informacijama o penjačkim lokacijama i smjerovima. 

Tradicionalno, glavni izvor informacija za penjače koji žele penjati na stijenama u prirodi su tiskani penjački vodiči. Ovi vodiči sadrže detaljne opise penjačkih lokacija, karte pristupa, kao i skicirane prikaze stijene ili često nazivane \textit{topo} s ocrtanim linijama penjačkih smjerova, njihovim nazivima i težinama. Iako su desetljećima bili nezamjenjiv alat, tiskani vodiči imaju ograničenja. Neki od ključnih nedostataka su statičnost i zastarijevanje podatka, nepraktičnost nošenja, nekonzistentnost između različitih izdanja te, najvažnije, poteškoće u interpretaciji dvodimenzionalnih skica na stvarnoj, trodimenzionalnoj stijeni zbog liminitiranog broja slika koji se mogu staviti u vodič.

S pojavom interneta i pametnih telefona, razvile su se digitalne platforme i mobilne aplikacije koje su djelomično riješile problem dostupnosti i ažurnosti podataka. One omogućuju centralizirano prikupljanje informacija, korisničke komentare i lakšu pretragu. Osim toga, nude i napredne funkcionalnosti poput vođenja osobnog dnevnika uspona, analize statistike, praćenja napretka i povezivanja s drugim penjačima.
Unatoč tim prednostima, digitalna rješenja nisu bez nedostataka. Unatoč tim prednostima, i digitalna rješenja imaju svoja ograničenja. Ključni nedostaci uključuju oslananje na statične, dvodimenzionalne prikaze koji ne rješavaju problem interpretacije na terenu, kao i praktične izazove poput ovisnosti o trajanju baterije i dostupnosti internetskog signala na udaljenim lokacijama.

Navedeni nedostaci postojećih alata stvaraju potrebu za rješenjem koje pokriva njihove nedostatke. Cilj je iskoristiti mobilnu tehnologiju kako bi se stvorilo rješenje koje bi minimiziralo navedene nedostatke. Ideja je omogućiti penjačima da jednostavnim usmjeravanjem kamere mobilnog uređaja prema stijeni dobije vizualnu informaciju o položaju i nazivima smjerova izravno u stvarnom okruženju korištenjem tehnologije proširene stvarnosti. Takav pristup ne samo da štedi vrijeme i smanjuje frustracije, već i omogućuje sigurnije iskustvo penjanja. 