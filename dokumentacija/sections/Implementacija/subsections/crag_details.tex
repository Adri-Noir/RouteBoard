\section{Detalji penjačke lokacije}

Odabirom penjališta iz pretrage, s geografkse karte ili drugih pregleda, korisnik pristupa zaslonu s detaljnim informacijama o penjačkoj lokaciji. Zaslon je podijeljen na nekoliko cjelina. Na vrhu se nalazi istaknuta fotografija penjališta, zajedno s njegovim nazivom i osnovnim podacima o broju sektora i penjačkih smjerova. Na web aplikaciji se nalazi isti prikaz, ali s manjim promjenama. Slika penjačke lokacije nalazi se ispod geografske karte i broj sektora i penjačkih smjerova nalazi se u prikazu sektora i penjačkih smjerova.

\begin{figure}[H]
    \centering
    \begin{subfigure}[b]{\textwidth}
        \centering
        \includegraphics[width=0.3\textwidth]{images/implementacija/crag-details/crag-details-top.png}
        \caption{Mobilna aplikacija}
        \label{fig:detalji_penjalista_mob}
    \end{subfigure}
    \hfill
    \begin{subfigure}[b]{\textwidth}
        \centering
        \includegraphics[width=0.9\textwidth]{images/implementacija/web/crag-details/crag-details-top.png}
        \caption{Web aplikacija}
        \label{fig:detalji_penjalista_web}
    \end{subfigure}
    \caption{Detalji penjačke lokacije na mobilnoj i web aplikaciji}
    \label{fig:detalji_penjališta_1}
\end{figure}

Odmah ispod, nalazi se komponenta s vremenskom prognozom. Ona prikazuje trenutne vremenske uvjete, kao i detaljnu prognozu po satima za sljedećih 14 dana. Detaljna prognoza uključuje temperaturu, vjerojatnost i količina padalina, brzinu vjetra, UV indeks i ostale relevantne podatke. Ova funkcionalnost je važna za planiranje penjačkih izleta.

\begin{figure}[H]
    \centering
    \begin{subfigure}[b]{\textwidth}
        \centering
        \includegraphics[width=0.35\textwidth]{images/implementacija/crag-details/crag-weather-1.png}
        \caption{Mobilna aplikacija}
        \label{fig:vremenska_prognoza_mob}
    \end{subfigure}
    \hfill
    \begin{subfigure}[b]{\textwidth}
        \centering
        \includegraphics[width=0.9\textwidth]{images/implementacija/web/crag-details/crag-weather.png}
        \caption{Web aplikacija}
        \label{fig:vremenska_prognoza_web}
    \end{subfigure}
    \caption{Vremenska prognoza na mobilnoj i web aplikaciji}
    \label{fig:vremenska_prognoza}
\end{figure}


Sljedi interaktivna karta penjališta koja prikazuje precizne lokacije svih sektora, omogućujući korisniku lako snalaženje i planiranje kretanja između njih. 

\begin{figure}[H]
    \centering
    \begin{subfigure}[b]{\textwidth}
        \centering
        \includegraphics[width=0.35\textwidth]{images/implementacija/crag-details/crag-map.png}
        \caption{Mobilna aplikacija}
        \label{fig:interaktivna_karta_mob}
    \end{subfigure}
    \hfill
    \begin{subfigure}[b]{\textwidth}
        \centering
        \includegraphics[width=0.9\textwidth]{images/implementacija/web/crag-details/crag-map.png}
        \caption{Web aplikacija}
        \label{fig:interaktivna_karta_web}
    \end{subfigure}
    \caption{Interaktivna karta penjališta na mobilnoj i web aplikaciji}
    \label{fig:interaktivna_karta}
\end{figure}

Ispod karte nalazi se prikaz detalja za sektore. Prije ikakvih odabira, korisniku se prikazuje popis svih penjačkih smjerova grupirano po sektorima te distribucija težina na penjačkoj lokaciji. Klikom na određenu težinu u grafu filtriraju se svi penjački smjerovi koji pripadaju odabranoj težini. Lista penjačkih smjerova na mobilnoj aplikaciji je horizontalna lista koja prikazuje penjački smjer u većem formatu kako bi se mogla bolje vidjeti slika penjačkog smjera. Preko slike nalaze se detalji o penjačkom smjeru poput naziva, težine, dužine, tipa i broju uspona.

\begin{figure}[H]
    \centering
    \begin{subfigure}[b]{\textwidth}
        \centering
        \includegraphics[width=0.35\textwidth]{images/implementacija/crag-details/crag-all-sectors-tabs.png}
        \caption{Mobilna aplikacija}
        \label{fig:prikaz_detalja_za_sektore_mob}
    \end{subfigure}
    \hfill
    \begin{subfigure}[b]{\textwidth}
        \centering
        \includegraphics[width=0.9\textwidth]{images/implementacija/web/crag-details/crag-all-sectors.png}
        \caption{Web aplikacija}
        \label{fig:prikaz_detalja_za_sektore_web}
    \end{subfigure}
    \caption{Prikaz detalja za sektore}
    \label{fig:prikaz_detalja_za_sektore}
\end{figure}

Ako korisnik preferira tablični prikaz na mobilnoj aplikaciji, može to promijeniti klikom na gumb pored naslova "Sektori" (eng. \textit{Sectors}). Na web aplikaciji tablični prikaz je zadani prikaz. Tablični prikaz omogućuje pregled svih penjačkih smjerova te sortiranje po nazivu, težini, dužini, tipu te broju uspona. S obzirom da na web aplikaciji nije grupiran prikaz po sektorima, postoje filteri koji korisnik može koristiti za filtriranje penjačkih smjerova po imenu, tipu i sektoru.

\begin{figure}[H]
    \centering
    \includegraphics[width=0.35\textwidth]{images/implementacija/crag-details/crag-all-sectors-table.png}
    \caption{Tablični prikaz sektora na mobilnoj aplikaciji}
    \label{fig:tablični_prikaz_sektora}
\end{figure}

Klikom na izbornik "Svi sektori" (eng. \textit{All sectors}) korisniku prikazuje se popis svih sektora na penjačkoj lokaciji. Odabirom određenog sektora, korisniku prikazuje se graf težina filtriran po odabranom sektoru, slike sektora te popis svih penjačkih smjerova u tom sektoru. Klikom na određeni stupac na grafu težina, popis penjačkih smjerova se filtrira po označenoj težini. Tablični prikaz je isto dostupan klikom na gumb pored naslova "Sektori" (eng. \textit{Sectors}). Na web aplikaciji nalaze se ista funkcionalnost, ali s različitim prikazom.

\begin{figure}[H]
    \centering
    \begin{subfigure}[b]{0.35\textwidth}
        \centering
        \includegraphics[width=\textwidth]{images/implementacija/crag-details/crag-selected-sector.png}
        \caption{Mobilna aplikacija}
        \label{fig:prikaz_detalja_za_odabrani_sektor_mob}
    \end{subfigure}
    \hfill
    \begin{subfigure}[b]{0.6\textwidth}
        \centering
        \includegraphics[width=1\textwidth]{images/implementacija/web/crag-details/crag-selected-sector.png}
        \caption{Web aplikacija}
        \label{fig:prikaz_detalja_za_odabrani_sektor_web}
    \end{subfigure}
    \caption{Prikaz detalja za odabrani sektor}
    \label{fig:prikaz_detalja_za_odabrani_sektor}
\end{figure}

